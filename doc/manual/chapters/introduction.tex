\chapter{Introduction}
\label{introduction}

%------------------------------------------------------------------------------
\section{The FEniCS project}

\fixme{Automation of CMM, other components of \fenics{}}


Let $\{V_i\}_{i=1}^r$ be a given set of discrete function
spaces defined on a triangulation $\mathcal{T}$ of $\Omega \subset
\R^d$ and consider a general multilinear form $a$ defined on the
product space $V_1 \times V_2 \times \cdots \times V_r$:
\begin{equation}
  a : V_1 \times V_2 \times \cdots \times V_r \rightarrow \R.
\end{equation}
Typically, $r = 1$ (linear form) or $r = 2$ (bilinear form), but
\ffc{} can handle multilinear forms of arbitrary arity $r$.

Let now
$\{\varphi_i^1\}_{i=1}^{M_1},
 \{\varphi_i^2\}_{i=1}^{M_2}, \ldots,
 \{\varphi_i^r\}_{i=1}^{M_r}$
be bases of $V_1, V_2, \ldots, V_r$ and let $i = (i_1, i_2, \ldots,
i_r)$ be a multiindex. The multilinear form $a$ then
defines a rank $r$ tensor given by
\begin{equation}
  A_i = a(\varphi_{i_1}^1, \varphi_{i_2}^2, \ldots, \varphi_{i_r}^r).
\end{equation}
In the case of a bilinear form, the tensor $A$ is a matrix (the
stiffness matrix), and in the case of a linear form, the tensor $A$ is
a vector (the load vector).


%------------------------------------------------------------------------------
\section{Overview}
