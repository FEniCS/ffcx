\chapter{Quickstart}
\index{quickstart}

This chapter demonstrates how to get started with \ffc{}, including
downloading and installing the latest version of \ffc{}, and compiling
Poisson's equation. These topics are discussed in more detail
elsewhere in this manual. In particular, see
Appendix~\ref{app:installation} for detailed installation instructions
and Chapter~\ref{sec:formlanguage} for a detailed discussion of the
form language.

%------------------------------------------------------------------------------
\section{Downloading and installing \ffc{}}
\index{downloading}
\index{installation}

The latest version of \ffc{} can be found on the \fenics{} web page:
\begin{code}
 http://www.fenics.org/
\end{code}
The following commands illustrate the installation process, assuming
that you have downloaded release 0.1.0 of \ffc{}:
\begin{code}
 # tar zxfv ffc-0.1.0.tar.gz
 # cd ffc-0.1.0
 # python setup.py install
\end{code}
Make sure that you download the latest release (which is note 0.1.0).

Note that you may need to be root on your system to do the last
step. You may also need to install the Python packages \fiat{} and
Numeric. (See Appendix~\ref{app:installation} for detailed instructions.)
\index{FIAT}
\index{Numeric}

%------------------------------------------------------------------------------
\section{Compiling Poisson's equation with \ffc{}}
\index{Poisson's equation}

The discrete variational (finite element) formulation of Poisson's
equation, $-\Delta u = f$, reads: Find $U \in V_h$ such that
\begin{equation} \label{eq:varform}
  a(v, U) = L(v) \quad \forall v\in \hat{V}_h, 
\end{equation}
with $(\hat{V}_h, V_h)$ a pair of suitable function spaces (the test and
trial spaces). The bilinear form $a : \hat{V}_h \times V_h \rightarrow
\R$ is given by
\begin{equation}
  a(v, U) = \int_{\Omega} \nabla v \cdot \nabla U \dx
\end{equation}
and the linear form $L : \hat{V}_h \rightarrow \R$ is given by
\begin{equation}
  L(v) = \int_{\Omega} v \, f \dx.
\end{equation}

To compile the pair of forms $(a, L)$ into code that can called to
assemble the linear system $A x = b$ corresponding to the variational
problem (\ref{eq:varform}) for a pair of discrete function spaces,
specify the forms in a text file with extension \texttt{.form},
e.g. \texttt{Poisson.form}, as follows:
\begin{code}
 element = FiniteElement(``Lagrange'', ``triangle'', 1)

 v = BasisFunction(element)
 U = BasisFunction(element)
 f = Function(element)
  
 a = dot(grad(v), grad(U))*dx
 L = f*v*dx
\end{code}

The example is given for piecewise linear finite elements in two
dimensions, but other choices are available, including arbitrary order
Lagrange elements in two and three dimensions.

To compile the pair of forms implemented in the file
\texttt{Poisson.form}, call the compiler on the command-line as
follows:
\begin{code}
 # ffc Poisson.form
\end{code}
This generates the file \texttt{Poisson.h} which implements the forms
in C++ for inclusion in \dolfin{}. For help on the \texttt{ffc}
command, including compilation for other systems than \dolfin{}, type
\texttt{ffc -h} or \texttt{man ffc}.
\index{ffc}
\index{man page}
