\chapter{Installation}
\label{app:installation}
\index{installation}

\newcommand{\ufc}{UFC}

The \ufc{} package consists of two parts, the main part being a single header file called ufc.h.
In addition we have provided a set of Python utilities to simplify the generation of \ufc{} compliant code.

%The package is portable and should work on any system with a standard Python installation.
Questions, bug reports and patches concerning
the installation should be directed to the
\ufc{} mailing list at the address
\begin{code}
ufc-dev@fenics.org
\end{code}

\ufc{} must currently be installed directly from source, but Debian
(Ubuntu) packages will be available in the future, for \ufc{} and
other \fenics{} components.

%------------------------------------------------------------------------------

\section{Installing from source}

\subsection{Dependencies and requirements}
\index{dependencies}

The installation system is based on distutils, and should work on any system with a standard Python installation.

\subsubsection{Installing Python}

If Python is not installed on your system, it can be downloaded from
\begin{code}
http://www.python.org/
\end{code}
Follow the installation instructions for Python given on the Python web page.
For Debian (Ubuntu) users, the package to install is named \texttt{python}.

\subsection{Installing \ufc{}}

\ufc{} follows the standard installation procedure for Python
packages. Enter the source directory of \ufc{} and issue the
following command:
\begin{code}
# python setup.py install
\end{code}
This will install the \ufc{} utility Python package in a subdirectory
called \texttt{ufc} in the default location for user-installed Python
packages (usually something like \texttt{/usr/lib/python2.5/site-packages}).
In addition, the compiler executable (a Python script) will be
installed in the default directory for user-installed Python scripts
(usually in \texttt{/usr/bin}).

To see a list of optional parameters to the installation script, type
\begin{code}
# python setup.py install --help
\end{code}
If you don't have root access to the system you are using, you can
pass the \texttt{--home} option to the installation script to install
\ufc{} in your home directory:
\begin{code}
# mkdir ~/local
# python setup.py install --home ~/local
\end{code}
This installs the \ufc{} package in the directory \texttt{\~{}/local/lib/python}.
If you use this option, make sure to set the environment variable \texttt{PYTHONPATH}
to \texttt{\~{}/local/lib/python} and to add \texttt{\~{}/local/bin}
to the \texttt{PATH} environment variable.


%------------------------------------------------------------------------------
\section{Debian (Ubuntu) package}
\index{Debian package}
%\index{Ubuntu package}

Debian packages for simpler installation of \fenics{} packages are in preparation and will be available in the future.

